\documentclass[notheorems,mathserif,table,compress]{beamer}  %dvipdfm选项是关键,否则编译统统通不过
%%------------------------常用宏包------------------------
%%注意, beamer 会默认使用下列宏包: amsthm, graphicx, hyperref, color, xcolor, 等等
\usepackage{fontspec,xunicode,xltxtra}  % for XeTeX
\usepackage{comment}
\usepackage{fancybox}
 \usepackage{enumerate}
\usepackage{color}

%%------------------------ThemeColorFont------------------------
%% Presentation Themes
% \usetheme[<options>]{<name list>}
\usetheme{Madrid}
%% Inner Themes
% \useinnertheme[<options>]{<name>}
%% Outer Themes
% \useoutertheme[<options>]{<name>}
\useoutertheme{miniframes} 
%% Color Themes 
% \usecolortheme[<options>]{<name list>}
%% Font Themes
% \usefonttheme[<options>]{<name>}
\setbeamertemplate{background canvas}[vertical shading][bottom=white,top=structure.fg!7] %%背景色, 上25%的蓝, 过渡到下白.
\setbeamertemplate{theorems}[numbered]
\setbeamertemplate{navigation symbols}{}   %% 去掉页面下方默认的导航条.
\usepackage{zhfontcfg}
\usepackage{wrapfig}
%\setsansfont[Mapping=tex-text]{文泉驿正黑}  %% 需要fontspec宏包
     %如果装了Adobe Acrobat,可在font.conf中配置Adobe字体的路径以使用其中文字体
     %也可直接使用系统中的中文字体如SimSun,SimHei,微软雅黑 等
     %原来beamer用的字体是sans family;注意Mapping的大小写,不能写错
     %设置字体时也可以直接用字体名,以下三种方式等同:
     %\setromanfont[BoldFont={黑体}]{宋体}
     %\setromanfont[BoldFont={SimHei}]{SimSun}
     %\setromanfont[BoldFont={"[simhei.ttf]"}]{"[simsun.ttc]"}
%%------------------------MISC------------------------
\graphicspath{{figures/}}         %% 图片路径. 本文的图片都放在这个文件夹里了.
%%------------------------正文------------------------
\begin{document}
\XeTeXlinebreaklocale "zh"         % 表示用中文的断行
\XeTeXlinebreakskip = 0pt plus 1pt % 多一点调整的空间
%%----------------------------------------------------------
%% This is only inserted into the PDF information catalog. Can be left
%% out.
%%%
%% Delete this, if you do not want the table of contents to pop up at
%% the beginning of each subsection:
\begin{comment}
\AtBeginSection[]{                              % 在每个Section前都会加入的Frame
  \frame<handout:0>{
    \frametitle{Content}\small
    \tableofcontents[current,currentsubsection]
  }
}
\AtBeginSubsection[]                            % 在每个子段落之前
{
  \frame<handout:0>                             % handout:0 表示只在手稿中出现
  {
    \frametitle{下一节内容}\small
    \tableofcontents[current,currentsubsection] % 显示在目录中加亮的当前章节
  }
}
\end{comment}
%%----------------------------------------------------------
\title[Image Segmentation]{Image Segmentation}
\author[赵海伟\ 王如晨\ 戴嘉伦 ]{\textcolor{olive}{赵海伟\ 王如晨\  戴嘉伦 }}
  %\hspace{2.28em}导师~~\textcolor{olive}{姬光荣}~教授}
\institute[CVBIOUC]{\small\textcolor{violet}{CVBIOUC}}
\date{\today}
%\titlegraphic{\vspace{-6em}\includegraphics[height=7cm]{ouc}\vspace{-6em}}
\frame{ \titlepage }
%%----------------------------------------------------------
%\section*{目录}
%\frame{\frametitle{目录}\tableofcontents}
%%----------------------------------------------------------

%\section{Beamer类和XeTeX概览} %如果你想书签不出现问题,请不要用\XeTeX
                                 %这类复杂的指令,直接写XeTeX吧
\section{Image Segmentation}
\begin{frame}
   \frametitle{Image Segmentation}
   \begin{enumerate}
   \item  {\textbf{\Large Introduction}}\newline
    \item {\textbf{\Large Methods}} 
    \end{enumerate}
\end{frame}


\subsection{Introduction}
\begin{frame}
    \frametitle{Introduction}
    \textbf{Image segmentation:} 
	\begin{itemize}
	\item The process of partitioning a digital image into \textbf{multiple segments}.\newline
	\end{itemize}
    \textbf{The goal of segmentation:}
	\begin{itemize}
	\item \textbf{Simplify} and \textbf{change} the representation of an image into something that is more meaningful and easier to analyze 
	\end{itemize}
\end{frame}


\begin{frame}
    \frametitle{Introduction}
   \begin{figure}
   \begin{minipage}[t]{0.4\textwidth}
   \centering
   \includegraphics[width=1.8in]{tiger1.png}
   \end{minipage}
   \begin{minipage}[t]{0.4\textwidth}
   \centering
   \includegraphics[width=1.8in]{tiger2.png}
   \end{minipage} 
   \end{figure}
\end{frame}

\begin{frame}
\frametitle{Introduction}
   \begin{figure}[!ht]
   \begin{minipage}[t]{0.4\textwidth}
   \centering
   \includegraphics[width=1.8in]{algea.jpg}
   \end{minipage}
   \begin{minipage}[t]{0.4\textwidth}
   \centering
   \includegraphics[width=1.8in]{algea2.jpg}
   \end{minipage} 
   \end{figure} 
\end{frame}

\begin{frame}
    \frametitle{Introduction}
    %Pre-process $\Longrightarrow$ Object Detection $\Longrightarrow$ Image Segmentation $\Longrightarrow$ Image Recognition
    \begin{figure}[!ht]
    \centering
    \includegraphics[width=3.0in]{path.png}
    \end{figure} 
\end{frame}


\begin{comment}
\subsection{Applications}
\begin{frame}
    \frametitle{Applications}
    \begin{itemize}
    \item Computer Vison
    \item Content-based image retrieval
    \item Object Detection
    \item Recognition Tasks
    \item Traffic Control Systems
    \item Video surveillance
    \end{itemize}
%To be useful, these techniques must typically be combined with a domain's specific knowledge in order to effectively solve the domain's segmentation problems.
\end{frame}
\end{comment}



\subsection{Methods}


\begin{frame}
    \frametitle{Methods}
    Segmentation algorithms are generally based on \underline{\textbf{two basic properties}} of gray-scale values
    \begin{itemize}
    \item Discontinuity
    \item Similarity \newline
    \end{itemize}
   % \emph{\textbf{Choice of technique depends on peculiar characteristics of individual problems.}}\\
   %\definecolor{cyanlight}{rgb}{0.0, 0.72, 0.92}
   %\color{cyanlight}{text}
    {\color{blue}{\emph{\textbf{A universal algorithm of segmentation does not exist, as each type of image corresponds to a specific approach.}}}}
\end{frame}


\begin{frame}
    \frametitle{Methods}
    \begin{itemize}
    \item Thresholding
    \item Edge-based Segmentation
    \item Region-based Segmentation
    \item State-of-the-art Methods
    \end{itemize}
   % Choice of technique depends on peculiar characteristics of individual problems.//
 %   A universal algorithm of segmentation does not exist, as each type of image corresponds to a specific approach
\end{frame}






\begin{comment}
\section{Region-based Methods}
\begin{frame}
    \frametitle{Region-based Methods}
    \begin{itemize}
    \item \textbf{Region}
    \item \textbf{Region Growing}
    \item \textbf{Region Spliting and Merging}
    \item \textbf{Watershed}
    \end{itemize}
\end{frame}

\subsection{Region}
\begin{frame}
    \frametitle{Region}
%    \begin{itemize}
     \begin{columns}
     \begin{column}[c]{0.45\textwidth}
     \textbf{\Large Definition:}
        \begin{itemize}
	  \item[-] A group of connected pixels with \textbf{similar properties} \newline
         \end{itemize}
    \textbf{\Large Idea:}
          \begin{enumerate}
          \item[-] Similarity
          \item[-] Spatial Proximity
	 \end{enumerate}
    \end{column}

    \begin{column}[c]{0.55\textwidth}
    \begin{figure}[!ht]
    \begin{minipage}[t]{0.5\linewidth}
    \centering
    \includegraphics[width=1.8in]{region_1.png}
    \end{minipage}
    \begin{minipage}[t]{0.5\linewidth}
    \centering
    \includegraphics[width=1.8in]{region_2.png}
    \end{minipage}
    \end{figure}
    \end{column}
    \end{columns}
\end{frame}


\subsection{Region Growing}
\begin{frame}
    \frametitle{Region Growing}
    \begin{enumerate}[{\color{black}{\Large (A)}}]
    \item \textbf{\Large Idea:}
    \end{enumerate}
    	\begin{itemize}
    	\item[-] {\color{blue}{Seed}}: The regions are growing from seeds points.\\ \hspace{0.4in}The corresponding regions grow by appending those neighboring pixels to each seed points.
	\item[-] {\color{blue}{Pre-defined Criterion}}: It groups pixels or sub-regions into larger regions based on pre-defined criterion.
	\item[-] {\color{blue}{End Condition}}: The regions keep growing until meeting the end condition.
    	\end{itemize}
  \begin{figure}[!ht]
  \centering\includegraphics[width=4.0in]{region1.png}
  \end{figure} 
\end{frame}

\begin{frame}
    \frametitle{Region Growing}
    \begin{enumerate}[{\color{black}{\Large (B)}}]
    \item  \textbf{\Large Algorithm:}
    \end{enumerate}
  %  \textbf{\Large Algorithm:}
    \begin{itemize}
    \item[step1] Initially, the region $R$ needs to be extracted. The region $R$ only contains its seed point $p.$
    \item[step2] Initially, a queue $Q$ contains the boundary points of $R.$ $Q$ contains the 8-neighborhood or 4-neighborhood of the seed point $p$.
    \item[step3] While $Q$ is not empty:
    	\begin{itemize}
	\item[-] for each neighboring pixel $p*$ of $p$ in $Q:$
	     \begin{itemize}
	     \item[-] \emph{if}  $p*$ is similar to $p:$\\
	        \hspace{0.1in} {\color{blue}{-}} \hspace{0.05in} $p*$ is added to $R,$ $p*$ is marked with a label.\\
        	\hspace{0.1in} {\color{blue}{-}} \hspace{0.05in} neighboring pixels of $p*$ (not in $R$) are added to $Q.$
	     \item[-] \emph{else} set $p*$ as non-similar.
	     \end{itemize}
	\end{itemize}	
    \end{itemize}
\end{frame}

\begin{frame}
  \frametitle{Region Growing}
  \begin{figure}[!ht]
  \centering\includegraphics[width=1.5in]{grow_algorithm.png}
  \end{figure} 
\end{frame}

\begin{frame}
\begin{figure}[!ht]
  \begin{minipage}[t]{0.3\textwidth}
  \centering
  \includegraphics[width=1.4in]{seed1.png}
  \end{minipage}
  \begin{minipage}[t]{0.3\textwidth}
  \centering
  \includegraphics[width=1.3in]{seed2.png}
  \end{minipage}  
  \begin{minipage}[t]{0.3\textwidth}
  \centering
  \includegraphics[width=1.4in]{seed3.png}
  \end{minipage}  
\end{figure} 
\end{frame}

\begin{frame}
    \frametitle{Region Growing}
    \begin{enumerate}[{\color{black}{\Large (C)}}]
    \item  \textbf{\Large Advantages:}
    \end{enumerate}
 %   \textbf{\Large Advantages:}\\
        \begin{itemize}
        \item Fast
        \item Simple conceptually
        \end{itemize}
    \begin{enumerate}[]
    \item  \hspace{0.25in}\textbf{\Large Disadvantages:}
    \end{enumerate}
  %  \textbf{\Large Disadvantages:}\\
        \begin{itemize}
%        \item Local method: no global view of the problem
        \item Dependent on seed point and pre-defined criterion
        \item Sensitive to noise
    \end{itemize}
\end{frame}




\subsection{Region Splitting and Merging}
\begin{frame}
    \frametitle{Region Spliting and Merging}
    \begin{enumerate}[{\color{black}{\Large (A)}}]
    \item  \textbf{\Large Idea:}
    \end{enumerate}
 %   \textbf{\Large Idea:}
    \begin{itemize}
    \item[-] {\color{blue}{Splitting}}: Subdivide the whole image into subsidiary regions recursively while a condition of homogeneity is not satisfied.%It starts with the whole image as a single region and subdivides it into subsidiary regions recursively while a condition of homogeneity is not satisfied.
    \end{itemize}
    \begin{itemize}
    \item[-] {\color{blue}{Merging}}: It starts with small regions and merges the regions that have similar characteristics to avoid over-segmentation.%It is the opposite of region splitting and works as a way of avoiding over-segmentation. It starts with small regions and merge the regions that have similar characteristics 
    \end{itemize}
  \begin{figure}[!ht]
  \centering\includegraphics[width=3.5in]{tree3.png}
  \end{figure}
\end{frame}


\begin{frame}
    \frametitle{Region Splitting and Merging}
    \begin{enumerate}[{\color{black}{\Large (B)}}]
    \item  \textbf{\Large Algorithm:}
    \end{enumerate}
  %  \textbf{\Large Algorithm:}\\
    \begin{itemize}
    \item[step1] {\emph{If}} a region $R$ is inhomogeneous ($P_1(R)=FALSE$), {\emph{then}} $R$ is split into four sub-regions.
    \item[step2] {\emph{If}} two adjacent regions $R_i$ and $R_j$ are homogeneous ($P_2(R_i \cup R_j)=TRUE$), {\emph{then}} they are  merged.
    \item[step3] The algorithm stops when no further splitting or merging is possible.
  \begin{figure}[!ht]
  \centering\includegraphics[width=3.5in]{split.png}
  \end{figure} 
    \end{itemize}
\end{frame}

\begin{frame}
  \frametitle{Region Splitting and Merging}
  \begin{figure}[!ht]
  \centering\includegraphics[width=3.5in]{tree2.png}
  \caption{$T$=1}
  \end{figure} 
\end{frame}




\begin{frame}
  \frametitle{Region Splitting and Merging}
  \begin{figure}[!ht]
  \centering\includegraphics[width=3.5in]{region2.png}
  \end{figure} 
\end{frame}

\begin{frame}
    \frametitle{Region Splitting and Merging}
    \begin{enumerate}[{\color{black}{\Large (C)}}]
    \item  \textbf{\Large Advantages:}
    \end{enumerate}
%    \textbf{\Large Advantages:}\\
        \begin{itemize}
        %\item 
	\item Applicability in complex scenarios
    %    \item Simple conceptually
        \end{itemize}
    \begin{enumerate}[]
    \item  \hspace{0.25in}\textbf{\Large Disadvantages:}
    \end{enumerate}
 %   \textbf{\Large Disadvantages:}\\
        \begin{itemize}
        \item Cost of time and calculation
        \item Effect on boundaries of regions
    %    \item Dependent on seed point and pre-defined criteria
   %     \item Sensitive to noise
    \end{itemize}
\end{frame}


\subsection{Watershed}
\begin{frame}
    \frametitle{Watershed}
    \begin{enumerate}[{\color{black}{\Large (A)}}]
    \item  \textbf{\Large Idea:}
    \end{enumerate}
    \begin{itemize}
        \item[-] {\color{blue}{Hole}}: Image that a hole is done through each local minimum.\\ 
	\hspace{0.4in}The entire topography is flooded with water rising through the holes at a uniform rate.
        \item[-] {\color{blue}{Dam}}: When rising water in adjacent catchment basins is about the merge, a dam is built up to prevent merging.
        \item[-] {\color{blue}{Lines}}: These dam boundaries correspond to the watershed lines.
    \end{itemize}
  \begin{figure}[!ht]
  \centering\includegraphics[width=2.5in]{region3.png}
  \end{figure} 
\end{frame}



%\begin{frame}
%  \frametitle{Watershed}
%%  \begin{figure}[!ht]
%  \centering\includegraphics[width=3.5in]{region3.png}
%  \end{figure} 
%\end{frame}


\begin{frame}
  \frametitle{Watershed}
  \begin{figure}[!ht]
  \begin{minipage}[t]{0.6\textwidth}
  \centering
  \includegraphics[width=2.5in]{region4.png}
  \end{minipage}
  \begin{minipage}[t]{0.6\textwidth}
  \centering
  \includegraphics[width=2.5in]{region5.png}
  \end{minipage} 
  \end{figure}  
\end{frame}

\begin{frame}
    \frametitle{Watershed}
%    \textbf{Principle:}\\
%        \begin{itemize}
%        \item To prevent the merging of water from two catchment basins
%        \end{itemize}
    \begin{enumerate}[{\color{black}{\Large (B)}}]
    \item  \textbf{\Large Algorithm:}
    \end{enumerate}
%   \textbf{\Large Algorithm:}\\
        \begin{itemize}
        \item[step1] Start with all pixels with the {\textbf{lowest}} possible value. These pixels form the basis for initial watershed.
        \item[step2] For each group of pixels of intensity level $k:$ %($k$ means the difference between the neighboring pixels of regions and ):
	     \begin{itemize}
             \item[-] \emph{If} the pixels are adjacent to exactly {\textbf{one}} exising region, add these pixels to that region. 
             \item[-] \emph{Else if} the pixels are adjacent to {\textbf{more than one}} existing regions, marks boundary. 
	           \begin{itemize}
		   \item[-] \emph{Else} start a new region.
		   \end{itemize}
             \end{itemize}
  %      \item These dam boundaries correspond to the watershed lines
        \end{itemize}
\end{frame}


\begin{frame}
  \frametitle{Watershed}
  \begin{figure}[!ht]
  \begin{minipage}[t]{0.35\textwidth}
  \centering
  \includegraphics[width=1.5in]{water_origin.png}
  \end{minipage}
  \begin{minipage}[t]{0.35\textwidth}
  \centering
  \includegraphics[width=1.5in]{water_over.png}
  \end{minipage} 
  \end{figure}  
\end{frame}
%\begin{frame}
%    \frametitle{Watershed}
  %  \textbf{Underlying operation:}\\
%	\begin{itemize}
  %       \item \textbf{Binary} morphological dilation 
 %       \end{itemize}
 %   \textbf{Dam construction sub-algorithm:}\\
%	\begin{itemize}
%	\item Initially, the set of pixels with minimum gray level are 1, others are 0
%	\item In each subsequent step, we flood the 3D topography from below and the pixels covered by the rising water are 1s and others 0s
%	\item At flooding step n-1, there are two connected components. At flooding step n, there is only one connected component
%	\end{itemize}
%	\end{frame}


\begin{frame}
    \frametitle{Watershed}
%    \textbf{Principle:}\\
%        \begin{itemize}
%        \item To prevent the merging of water from two catchment basins
%        \end{itemize}
    \begin{enumerate}[{\color{black}{\Large (C)}}]
    \item  \textbf{\Large Improved Algorithm:}
    \end{enumerate}
%   \textbf{\Large Algorithm:}\\
        \begin{itemize}
        \item[step1] Initially, {\textbf{label some pixels}} in your interested regions manually. \\
	Start with 8-neighborhood or 4-neighborhood of the labeled pixels. 
        \item[step2] For each intensity level k:
	     \begin{itemize}
             \item[-] \emph{If} adjacent to exactly {\textbf{one}} exising region, add these pixels to that region. 
             \item[-] \emph{Else if} adjacent to {\textbf{more than one}} existing regions, marks boundary. 
	           \begin{itemize}
		   \item[-] \emph{Else} start a new region.
		   \end{itemize}
             \end{itemize}
  %      \item These dam boundaries correspond to the watershed lines
        \end{itemize}
\end{frame}


\begin{frame}
  \frametitle{Watershed}
  \begin{figure}[!ht]
  \begin{minipage}[t]{0.4\textwidth}
  \centering
  \includegraphics[width=1.5in]{watershed1.png}
  \end{minipage}
  \begin{minipage}[t]{0.4\textwidth}
  \centering
  \includegraphics[width=1.5in]{watershed3.png}
  \end{minipage} 
  \end{figure}  
\end{frame}
\end{comment}


\end{document}
